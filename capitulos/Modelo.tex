\section{Modelo}

Se aborda el modelado de un controlador predictivo no lineal para la grúa. El controlador se formula en términos de las variaciones de fuerzas $\Delta f_x(t)$ y $\Delta f_l(t)$, con discretización $T_s = 0.1$ s y horizonte de predicción $T = 1$ s.

\subsection{Modelo no lineal}

Se emplea una formulación con estados extendidos para implementar la acción integral en ACADO Toolkit:
\begin{equation}
\mathbf{x}_e(t)=
\begin{bmatrix}
x_m(t) \\ y_m(t) \\ \dot{\theta}(t) \\ x(t) \\ l(t) \\ \theta(t) \\ \dot{x}(t) \\ \dot{l}(t) \\ f_x(t) \\ f_l(t)
\end{bmatrix},
\qquad
\mathbf{u}_e(t)=
\begin{bmatrix}
\dot{f}_x(t) \\ \dot{f}_l(t)
\end{bmatrix}.
\end{equation}

El modelo implícito no lineal es:
\begin{equation}
\mathbf{0} = \mathbf{f}\big(\dot{\mathbf{x}}_e(t),\mathbf{x}_e(t),\mathbf{u}_e(t)\big),
\end{equation}

definido por:
\begin{equation}
    \begin{aligned}
        0 &= x_m - x - l\sin(\theta), \\
        0 &= y_m - l\cos(\theta), \\
        0 &= m l^2 \ddot{\theta} + m l \cos(\theta)\ddot{x}
            + 2 m l \dot{l}\dot{\theta} + m g l \sin(\theta), \\
        0 &= \dot{x} - \dot{x}, \\
        0 &= \dot{l} - \dot{l}, \\
        0 &= \dot{\theta} - \dot{\theta}, \\
        0 &= (M+m)\ddot{x} + m l \cos(\theta)\ddot{\theta}
            + m l \sin(\theta)\ddot{l}
            + b_x\dot{x}\\
            &\quad + 2 m \cos(\theta)\dot{l}\dot{\theta}
            - m l \sin(\theta)\dot{\theta}^2 - f_x, \\
        0 &= (M+m)\ddot{l} + m \sin(\theta)\ddot{x}
            + b_l\dot{l} - m l \dot{\theta}^2\\
            &\quad- m g \cos(\theta) - f_l, \\
        0 &= \dot{f}_x - \dot{f}_x, \\
        0 &= \dot{f}_l - \dot{f}_l.
    \end{aligned}
\end{equation}

En ACADO Toolkit:

DifferentialState xm ym dtheta x l theta dx dl fx fl;

junto con la correspondiente definición del sistema de ecuaciones diferenciales implícitas, tal y como se proporciona en la plantilla base de la práctica.

\subsection{Matrices de ponderación}

Dado que el objetivo principal es el seguimiento de las posiciones de la carga $(x_m(t),y_m(t))$, se asignan ponderaciones elevadas a estos estados, fijando $r_{x_m}=50$ y $r_{y_m}=10$, mientras que para reducir las oscilaciones del péndulo se impone una referencia nula sobre el ángulo $\theta(t)$ con una ponderación $r_\theta=10$.

La acción de control incremental se penaliza mediante $\alpha_{\Delta f_x}=\alpha_{\Delta f_l}=0.01$, definiéndose así las matrices de ponderación del estado y del estado terminal como,
\begin{equation}
\mathbf{W} \in \mathbb{R}^{12\times12},
\qquad
\mathbf{W}_N \in \mathbb{R}^{10\times10},
\end{equation}
con estructura diagonal y valores no nulos únicamente en los estados y acciones relevantes para el control.

\subsection{Restricciones}

El sistema está sujeto a las siguientes restricciones físicas sobre estados, fuerzas y variaciones de control:
\begin{equation}
    \begin{aligned}
        -4 &\leq x_m(t) \leq 4 \;[\mathrm{m}], \\
        0 &\leq y_m(t) \leq 2 \;[\mathrm{m}], \\
        -4 &\leq x(t) \leq 4 \;[\mathrm{m}], \\
        0 &\leq l(t) \leq 2 \;[\mathrm{m}], \\
        -\frac{\pi}{40} &\leq \theta(t) \leq \frac{\pi}{40  } \;[\mathrm{rad}], \\
        -50 &\leq f_x(t) \leq 50 \;[\mathrm{N}], \\
        -50 &\leq f_l(t) \leq 50 \;[\mathrm{N}], \\
        -10 &\leq \Delta f_x(t) \leq 10 \;[\mathrm{N/step}], \\
        -10 &\leq \Delta f_l(t) \leq 10 \;[\mathrm{N/step}].
    \end{aligned}
\end{equation}

Esto se implementaría en ACADO Toolkit mediante las instrucciones:
%\lstinputlisting[language=MATLAB, firstline=166, lastline=174]{codigo/ControladorNMPC_Grua_old.m}
\lstinputlisting[language=MATLAB, firstline=172, lastline=180]{codigo/ControladorNMPC_Grua.m}

\subsection{Referencia}

Se desea que la carga de la grúa siga una trayectoria sinusoidal en ambas coordenadas $(x_m(t),y_m(t))$, manteniendo simultáneamente el ángulo $\theta(t)$ próximo a cero durante todo el movimiento.

La referencia se genera mediante,
% \begin{verbatim}
% Nsim = 30/Ts;
% Freq1 = 0.05;
% Freq2 = 0.1;
% Ref = [zeros(1,20) sin(Freq1*[0:Nsim-1]);
%        ones(1,20) 1+0.3*sin(Freq2*[0:Nsim-1])];
% \end{verbatim}

\lstinputlisting[language=MATLAB, firstline=250, lastline=254]{codigo/ControladorNMPC_Grua.m}

Resultando en la siguiente trayectoria deseada:
\begin{figure}[H]
    \centering
    \begin{tikzpicture}
    \begin{axis}[
        width=0.96\linewidth,
        height=0.4\linewidth,
        scale only axis=false,
        title={Referencias del sistema},
        xlabel={t(s)},
        ylabel={$x_m$ (m)},
        grid=major,
        %ytick={0, 200, 400, 600, 800, 1000, 1200, 1400},
        xmin=00, % El eje X empieza en 0
        xmax=30,
        legend pos=south east
    ]
        \addplot[
            color=NavyBlue,
            mark=none
        ]
        table[
        x=t,
        y=refxm
        ]{datos/PL6_1.dat};
        %\addlegendentry{$ref_{x_m}$}
    \end{axis}
    \end{tikzpicture}
    \begin{tikzpicture}
    \begin{axis}[
        width=0.96\linewidth,
        height=0.4\linewidth,
        scale only axis=false,
        %title={Salidas del sistema},
        xlabel={t(s)},
        ylabel={$y_m$ (m)},
        grid=major,
        %ytick={0, 200, 400, 600, 800, 1000, 1200, 1400},
        xmin=00, % El eje X empieza en 0
        xmax=30,
        legend pos=south east
    ]
        \addplot[
            color=NavyBlue,
            mark=none
        ]
        table[
        x=t,
        y=refym
        ]{datos/PL6_1.dat};
        %\addlegendentry{$ref_{x_m}$}
    \end{axis}
    \end{tikzpicture}
    \caption{Trayectoria de referencia para la carga de la grúa}
    %\label{fig:Comparación de velocidad de salida $\omega_e$ con el modelo base}
\end{figure}%xm ym

% \lstinputlisting[language=MATLAB, firstline=242, lastline=246]{codigo/ControladorNMPC_Grua_old.m}

\subsection{Modelo de planta}

Para la simulación de la planta se emplean los integradores internos de ACADO Toolkit, utilizando la función precompilada \texttt{integrate\_grua}, generada automáticamente a partir del modelo ODE cuando se habilita la opción \texttt{EXPORT=1}.

La simulación de la dinámica de la grúa se realiza según,
% \lstinputlisting[language=MATLAB, firstline=394, lastline=401]{codigo/ControladorNMPC_Grua_old.m}
\lstinputlisting[language=MATLAB, firstline=410, lastline=417]{codigo/ControladorNMPC_Grua.m}

donde $X$ representa el estado actual y $DU$ la acción de control calculada por el NMPC en cada instante.