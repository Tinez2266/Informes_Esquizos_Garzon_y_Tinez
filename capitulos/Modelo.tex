% =========================
% capitulos/Modelo.tex
% =========================

\section{Modelo en Simulink y componentes}
\subsection{Bloques disponibles para simular un FIS}
La simulación de un FIS en Simulink se realiza principalmente con los siguientes bloques \cite{mathworks_simulate_fis_simulink}:
\begin{itemize}
    \item \textbf{Fuzzy Logic Controller:} evalúa un FIS (tipo 1 y, según configuración, tipo 2) y entrega la salida del controlador.
    \item \textbf{Fuzzy Logic Controller with Ruleviewer:} equivalente al anterior, pero permite visualizar durante la simulación la inferencia de reglas.
    \item \textbf{FIS Tree:} permite simular estructuras en árbol de FIS interconectados (útil en diseños jerárquicos).
\end{itemize}

\subsection{Modelo \texttt{sltank}}
El modelo \texttt{sltank} implementa un lazo de control donde el bloque \textit{Fuzzy Logic Controller} recibe dos señales relacionadas con el nivel del agua y genera una acción sobre la válvula. Para que el bloque funcione, el parámetro \textit{FIS name} debe referenciar un objeto FIS existente en el \textit{workspace} de MATLAB (por ejemplo, un \texttt{mamfis} llamado \texttt{tank}) \cite{mathworks_simulate_fis_simulink}.

\subsection{Modelo \texttt{sltankrule} (Ruleviewer)}
La variante \texttt{sltankrule} utiliza el bloque \textit{Fuzzy Logic Controller with Ruleviewer}. La estructura general del lazo es la misma, pero se añade la posibilidad de inspeccionar qué reglas se activan y con qué intensidad. Además, al pausar la simulación se pueden ajustar manualmente las entradas en el visor para observar el efecto sobre la inferencia y la salida del controlador \cite{mathworks_simulate_fis_simulink}.
