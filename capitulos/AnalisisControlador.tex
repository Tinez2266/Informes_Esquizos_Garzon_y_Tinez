% =========================
% !TeX root = ../Plantilla.tex
% capitulos/AnalisisControlador.tex
% =========================

\section{Diseño y análisis del controlador difuso}
\subsection{Variables del FIS}
En el ejemplo se emplean dos entradas y una salida:
\begin{itemize}
    \item \textbf{Entrada 1 (\texttt{level}):} error de nivel (diferencia entre referencia y nivel medido).
    \item \textbf{Entrada 2 (\texttt{rate}):} razón de cambio del nivel (tendencia: subiendo/bajando).
    \item \textbf{Salida (\texttt{valve}):} acción sobre la válvula (velocidad/corrección de apertura o cierre).
\end{itemize}
\subsection{Reglas implementadas del controlador difuso}
El diseño de reglas evoluciona desde una lógica simple basada únicamente en el error de nivel hacia una estrategia que incorpora la tendencia (\texttt{rate}) para mejorar la estabilidad. Las reglas finales son:
\begin{itemize}
    \item Si el nivel es bajo, abrir la válvula rápido.
    \item Si el nivel es alto, cerrar la válvula rápido.
    \item Si el nivel es correcto y está aumentando, cerrar la válvula lentamente.
    \item Si el nivel es correcto y está disminuyendo, abrir la válvula lentamente.
\end{itemize}
Esta configuración reduce significativamente las oscilaciones alrededor del setpoint al amortiguar la corrección cerca de la referencia, logrando un seguimiento estable sin comportamientos transitorios pronunciados.

\subsection{Opciones de simulación y trazabilidad (comentario breve)}
La documentación distingue dos modos de simulación para los bloques relacionados: \textit{Interpreted execution} y \textit{Code generation}. Asimismo, el bloque puede exponer señales internas del proceso de inferencia para depuración: entradas fuzzificadas, fuerza de disparo de reglas, salidas por regla y salida agregada. Estas señales permiten entender mejor la contribución de cada regla y localizar configuraciones conflictivas.

\subsection{Correspondencias con \texttt{evalfis} / \texttt{evalfisOptions}}
El ejemplo también describe una equivalencia conceptual entre parámetros/puertos del
