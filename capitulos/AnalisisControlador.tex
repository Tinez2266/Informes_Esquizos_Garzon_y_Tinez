% =========================
% capitulos/AnalisisControlador.tex
% =========================

\section{Diseño y análisis del controlador difuso}
\subsection{Variables del FIS}
En el ejemplo se emplean dos entradas y una salida \cite{mathworks_simulate_fis_simulink}:
\begin{itemize}
    \item \textbf{Entrada 1 (\texttt{level}):} error de nivel (diferencia entre referencia y nivel medido).
    \item \textbf{Entrada 2 (\texttt{rate}):} razón de cambio del nivel (tendencia: subiendo/bajando).
    \item \textbf{Salida (\texttt{valve}):} acción sobre la válvula (velocidad/corrección de apertura o cierre).
\end{itemize}

\subsection{Primer diseño de reglas (solo error de nivel)}
Como punto de partida, se definen reglas que dependen únicamente del error de nivel. La lógica es directa: si el nivel es correcto, no se modifica la válvula; si el nivel es bajo, se abre rápido; si el nivel es alto, se cierra rápido. Esta base inicial es útil por su simplicidad, pero puede ser insuficiente para amortiguar correctamente el sistema \cite{mathworks_simulate_fis_simulink}.

\subsection{Comportamiento observado: oscilaciones alrededor del setpoint}
Al simular con las reglas iniciales, el nivel puede oscilar alrededor de la referencia. La causa principal es que el controlador decide solo por error instantáneo y no tiene en cuenta la tendencia: cuando el nivel se aproxima al setpoint, la corrección puede llegar tarde o ser demasiado intensa, favoreciendo sobreimpulso y oscilación \cite{mathworks_simulate_fis_simulink}.

\subsection{Mejora de reglas incorporando la variable \texttt{rate}}
Para reducir las oscilaciones, se añaden reglas activas cuando el nivel está cerca del setpoint (\textit{okay}) y se usa \texttt{rate} para actuar de forma más suave:
\begin{itemize}
    \item Si el nivel es \textit{okay} y está aumentando (\texttt{rate} positivo), cerrar la válvula lentamente.
    \item Si el nivel es \textit{okay} y está disminuyendo (\texttt{rate} negativo), abrir la válvula lentamente.
\end{itemize}
Con esta modificación se mejora el amortiguamiento cerca de la referencia y el seguimiento resulta más estable (sin oscilaciones apreciables) \cite{mathworks_simulate_fis_simulink}.

\subsection{Opciones de simulación y trazabilidad (comentario breve)}
La documentación distingue dos modos de simulación para los bloques relacionados: \textit{Interpreted execution} y \textit{Code generation}. Asimismo, el bloque puede exponer señales internas del proceso de inferencia para depuración: entradas fuzzificadas, fuerza de disparo de reglas, salidas por regla y salida agregada. Estas señales permiten entender mejor la contribución de cada regla y localizar configuraciones conflictivas \cite{mathworks_simulate_fis_simulink}.

\subsection{Correspondencias con \texttt{evalfis} / \texttt{evalfisOptions}}
El ejemplo también describe una equivalencia conceptual entre parámetros/puertos del
