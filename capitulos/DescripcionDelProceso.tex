% =========================
% capitulos/DescripcionDelProceso.tex
% =========================

\section{Descripción del proceso}
El sistema a controlar es un tanque de agua con una dinámica no lineal. La entrada de control actúa sobre el caudal de entrada mediante una válvula, mientras que el caudal de salida depende de un parámetro geométrico (diámetro del tubo de salida, constante) y de la presión asociada al nivel de agua, que varía con el propio nivel. Esta dependencia hace que el comportamiento global no sea lineal, lo que motiva el uso de un controlador basado en reglas \cite{mathworks_simulate_fis_simulink}.

El objetivo de control es que el nivel del tanque siga una referencia (setpoint) ante cambios y perturbaciones, con una respuesta estable (sin oscilaciones sostenidas) y con un error estacionario reducido. El ejemplo de Simulink permite comparar una solución inicial simple (reglas basadas solo en error) con una mejora que incorpora la tendencia del nivel (derivada o razón de cambio) \cite{mathworks_simulate_fis_simulink}.
