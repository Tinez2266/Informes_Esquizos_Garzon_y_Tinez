% !TeX root = ../Plantilla.tex
% =========================
% capitulos/Conclusiones.tex
% =========================

\section{Conclusiones}
Se ha demostrado mediante experimentación que un controlador difuso puede estabilizar efectivamente el nivel de un tanque. El análisis comenzó con un conjunto mínimo de reglas basadas únicamente en el error de nivel. La incorporación de la variable de tendencia (rate) permitió suavizar la respuesta y reducir estas oscilaciones, validando la hipótesis de que la información sobre la tasa de cambio mejora la estabilidad del sistema.

Los resultados experimentales evidencian la efectividad del enfoque incremental: partir de una estrategia simple y refinarla progresivamente permite comprender cómo cada regla y variable lingüística impacta en el comportamiento del sistema. Se ha comprobado que esta metodología iterativa es más eficaz para obtener un controlador robusto que diseñar exhaustivamente desde el inicio, facilitando además el depurado y la interpretación de los resultados observados durante cada etapa del experimento.
